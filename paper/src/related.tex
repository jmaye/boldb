Existing driving behavior models in psychology are largely subjective and based
on self-report scales~\cite{ranney94models}. They are difficult to quantify
because they include many psychological aspects like motivation, or risk
assessment.

Many works in the intelligent vehicle
literature~\cite{donges78two,mcruer80human,hess90control,macadam81application}
focus on modeling the driver behavior via their steering behavior or road
tracking information or desired driver's path as source of behavior's
information. Other works recognize driver's intentions via Bayesian reasoning on
a complex input including the driver's current control actions and the traffic
environment surrounding them~\cite{oliver00graphical,liu01modeling}. To our
knowledge, there has been few research works that combine traffic scenario
recognition and action prediction in an on-line and unsupervised fashion.

Maye \emph{et al.}~\cite{maye10inferring} were able to infer an action from a
direction sign in an indoor environment with a semi-supervised approach
using vision and prerecorded robot actions. We extend this idea to outdoor,
remove any supervision, and predict vehicle actions in an on-line fashion.
Meyer \emph{et al.}~\cite{meyer09probabilistic} predicted traffic situations
using Hidden Markov Models (HMM). They however restricted their situations space
by modeling states with respect to surrounding vehicles (distance, speed,
bearing) and manually segment image sequences for initial estimates. In this
paper, we exclude any manual intervention in the process and use a more complete
set of variables for predicting states.
Other works  \cite{heracles10vision,pugeault10learning} make use of supervised 
offline classification methods for learning the relation between driving actions and 
visual features. The actions are manually annotated and discretized in the training phase.

Change-point detection is a well-known problem in statistics. The CUSUM detector
\cite{page54continuous} uses piece-wise segments of Gaussian mean with noise.
Our implementation is largely inspired on the statistical models presented
in~\cite{adams07bayesian,fearnhead07online} that are easily applicable to
conjugate-exponential models. Several other applications, specially in the field
of computer vision, have used change-point detection
\cite{zhai05general,cemgil05hybrid}. To the author best knowledge this paper
represents the first application in the field of learning driving behaviors.
