We want to estimate the posterior probability distribution over actions
conditioned on the current traffic situation and segment. To this end, we
closely follow the strategy of Section~\ref{sec:labeling}. To each of the
traffic situation model $M_i$ is associated an action model $A_i$. This action
model $A_i$ is itself a parametric distribution $p(\mathbf{a}_t\mid
\boldsymbol{\eta}_i)$ with parameter vector $\boldsymbol{\eta}_i$ from which we
can sample an acceleration $\mathbf{a_t}$. This acceleration represents a
possible action of the model $A_i$. We can thus formulate the following model
that we estimate and update at each time step:

\begin{equation}
\label{eqn:action}
p(\mathbf{a_t} \mid r_t, l_t)=\sum_{\mathbf{c}_t} p(\mathbf{c}_t)p(\mathbf{a}_t\mid\mathbf{c}_t,r_t,l_t,\boldsymbol{\psi}^{\mathbf{c}_t})
\end{equation}

The distribution of \eqref{eqn:action} is a Gaussian Mixture Model (GMM) and
we apply on-line estimation whenever we receive a new data point. For the same
traffic situation $M_i$, we are able to model several possible behaviors
with (\ref{eqn:action})corresponding to the different Gaussian components. For
instance, when we reach a traffic light, we might brake when the light is red
and continue when it is green. Moreover, a driver does not always brake or
accelerate exactly the same manner every time. Finally, our system can adapt
to new drivers. In the same fashion as for Sec.\ref{sec:labeling}, we augment
each particle with the distribution (\ref{eqn:action}).
