We want to estimate the posterior probability distribution over actions
conditioned on the current traffic situation. To this end, we closely follow the
strategy of Sec.\ref{sec:labeling}. To each of the traffic situation model $M_i$
is associated an action model $A_i$. This action model $A_i$ is itself a
parametric distribution $p(\mathbf{a}_t\mid\boldsymbol{\eta}_i)$ with parameter
vector $\boldsymbol{\eta}_i$ from which we can sample an acceleration
$\mathbf{a_t}$. This acceleration represents a possible action of the model
$A_i$. We can thus formulate the following model that we estimate and update
at each time step:

\begin{equation}
\label{eqn:action}
p(\mathbf{a_t} \mid x_t,\mathbf{z}_{1:t},r_t).
\end{equation}

The distribution of (\ref{eqn:action}) is a Gaussian Mixture Model (GMM) and
we apply on-line estimation whenever we receive a new data point. For the same
traffic situation $M_i$, we are able to model several possible behaviors
with (\ref{eqn:action})corresponding to the different Gaussian components. For
instance, when we reach a traffic light, we might brake when the light is red
and continue when it is green. Moreover, a driver does not always brake or
accelerate exactly the same manner every time. Finally, our system can adapt
to new drivers. In the same fashion as for Sec.\ref{sec:labeling}, we augment
each particle with the distribution (\ref{eqn:action}).

A possible practical use of our algorithm could be in action prediction in
urban settings to assist a car driver. To implement this, the most probable
current segment $r_t$ is first sampled from $p(r_t\mid \mathbf{z}_{1:t})$
(Sec.\ref{sec:motion}). From $r_t$, the most probable situation model is
sampled from $p(x_t^{(r)}\mid \mathbf{z}_{1:t},r_t)$ (Sec.\ref{sec:action}). And
finally, the most probable action is inferred from
$p(\mathbf{a_t} \mid x_t,\mathbf{z}_{1:t},r_t)$.
