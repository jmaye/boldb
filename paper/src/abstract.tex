This paper presents a novel self-supervised on-line learning method to
discover driving behaviors from data acquired with an inertial
measurement unit (IMU) and a camera. Both sensors where mounted in a
car that was driven by a human through a typical city environment with
intersections, pedestrian crossings and traffic lights. The presented
system extracts motion segments from the IMU data and relates them to
the visual cues obtained from the camera data. It employs a Bayesian
on-line estimation method to discover the motion segments based on
change-point detection and uses a Dirichlet Compound Multinomial (DCM)
model to represent the visual features extracted from the camera
images. By incorporating these visual cues into the on-line estimation
process, labels are computed that are equal for similar motion
segments. As a result, typical traffic situations such as braking
maneuvers in front of a red light can be identified
automatically. Furthermore, appropriate actions in form of observed
motion changes are associated to the discovered traffic
situations. The approach is evaluated on a real data set acquired in
the center of Zurich.
