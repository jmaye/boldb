This paper presents a novel self-supervised online learning method to
discover driving behaviors from data acquired with an inertial
measurement unit (IMU) and a camera. Both sensors where mounted in a
car that was driven by a human through a typical city environment with
intersections, pedestrian crossings and traffic lights. The presented
system extracts motion segments from the IMU data and relates them to
the visual cues obtained from the camera data. It employs a Bayesian
on-line estimation method to discover the motion segments based on
change-point detection and uses a Dirichlet Compound Multinomial (DCM)
model to represent the visual features extracted from the camera
images. By incorporating these visual cues into the on-line estimation
process, labels are computed that are equal for similar motion
segments. As a result, typical traffic situations such as braking
maneuvres in front of a red light can be identified
automatically. Furthermore, appropriate actions in form of observed
motion changes are associated to the discovered traffic
situations. The approach is evaluated on a real data set acquired in
the center of Zurich.

%One research area that has turned more and more into the focus of
%interest during the last years is the development of driver
%intelligent assistant systems. In particular, a very active topic is
%the design of human-friendly vehicle control systems, able to meet the
%driver specific behaviors. In this paper we address the problem of
%learning driver behaviors based on on camera images and inertial
%information (IMU). Our approach proposes a novel on-line and
%unsupervised way of learning driver's behavior in different traffic
%conditions.  Specifically we learn the relationship between vehicle
%motion and image streams.  In this paper we introduce a novel
%technique that uses a \textit{change-point} detection model to segment
%IMU and image streams into portions corresponding to
%behaviors. Change-point detection is the problem of detecting abrupt
%changes to the parameters of a statistical model. This is computed
%on-line and in an unsupervised manner. The results from experiments in
%populated urban environments demonstrate the robustness of the
%algorithm to correctly identify several driving behaviors.
