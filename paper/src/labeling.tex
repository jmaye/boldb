Our aim is to find a label for each segmented motion pattern. This label
represents a traffic situation, e.g., a stop or turn condition. Moreover, we are
interested in associating two different motion segments to the same label
whenever they depict the same traffic situation. In the following, we show how
we can integrate this labeling into the on-line framework of
Section~\ref{sec:motion}.

\subsection{Traffic Situation Model}
As shown above, we denote the label of a segment $r_t$ as $l_t$. This label can
take values in $[1,2,\dots,N]$ corresponding to $N$ parametric models $M_1,M_2,
\dots,M_N$. Each of the $M_i$ is a generative model $p(\mathbf{c}_t\mid
\boldsymbol{\eta}_i)$ for a particular traffic situation with parameter vector
$\boldsymbol{\eta}_i$. At time $t$, we estimate the distribution over the known
models conditioned on the data seen so far and the segment we are in with the
product rule as

\begin{eqnarray}
\label{eqn:labeling}
p(l_t\mid r_t,\mathbf{c}_{1:t})&\propto&p(\mathbf{c}_{1:t}\mid l_t,r_t)
p(l_t\mid r_t)\nonumber\\
&=& p(\mathbf{c}^{r_t}\mid l_t,r_t)p(l_t\mid r_t),
\end{eqnarray}

where $\mathbf{c}^{r_t}$ represents data on the current segment $r_t$.

For the prior part in \eqref{eqn:labeling}, we use the posterior of the previous
time step, that is
$p(l_t\mid r_t)=p(l_{t-1}\mid r_{t-1},\mathbf{c}^{r_{t-1}})$. If we are in a new
segment with $r_t=0$, we set the prior to a uniform distribution over the known
models. The likelihood part in \eqref{eqn:labeling} is computed with the model
probability density function $p(\mathbf{c}_t\mid \boldsymbol{\eta}_i)$ in the
same fashion as in \eqref{eqn:preddistr}, i.e., using a conjugate prior with
hyperparameters $\boldsymbol{\psi}_i$ as will be detailed below.

As we want to be able to discover new traffic situations on-line, we have to
state if the current data $\mathbf{c}_t$ is unlikely to come from any of the
$N$ known models so far. We use statistical hypothesis testing for this. At each
time step, we thus perform $N$ statistical test. The likelihood ratio test is
suitable for our purpose. We compare model $M_i$ with a model learned over
segment $r_t$. The test statistic is

\begin{equation}
\label{eqn:statistic}
D = -2\ln\frac{p(\mathbf{c}_t\mid \boldsymbol{\psi}_i)}{p(\mathbf{c}_t\mid
\boldsymbol{\psi}_{ml})},
\end{equation}

where $\boldsymbol{\psi}_{ml}$ is the maximum likelihood solution for
$\boldsymbol{\psi}$ over the current segment $r_t$.

Based on the value of $D$, we have to decide whether to accept or reject the
null hypothesis, i.e., data in segment $r_t$ arises from model $M_i$. It is
shown that when the sample size grows towards infinity, $D$ converges towards a
Chi-square distribution with $K-1$ degrees of freedom, where $K$ is the
dimension of $\boldsymbol{\psi}$. In our case, the sample size represents the
number of data used to compute the maximum likelihood solution. As will be
shown below, it will be sufficient to approximate $D$ with a Chi-square
distribution. We will thus compare $D$ to the Chi-square value corresponding to
the desired statistical significance $\xi$ of the null hypothesis. In practice,
this is done by computing the value of the cumulative distribution
function of the Chi-square distribution for $D$ and reject the model
$M_i$ if it is below $\xi$.

In case all models are rejected, we create a new instance $M_{N+1}$
with hyperparameter vector $\boldsymbol{\psi}_{ml}$, set $p(l_t=N+1\mid r_t,
\mathbf{c}^{r_t})=p_{new}$, and $p(l_t=1:N\mid r_t,\mathbf{c}^{r_t})=
(1-p_{new})/N$. We update the hyperparameters $\boldsymbol{\psi}_i$ of model
$M_i$, such that $i=\argmax_{j=1:N}p(l_t=j\mid
r_t,\mathbf{c}^{r_t})$, with $\mathbf{c}_t$.

From an implementation point of view, we attach the distribution
\eqref{eqn:labeling}, the hyperparameters $\boldsymbol{\psi}_{ml}$, and the
incremental set of known models $M_i$ to each particle.

\subsection{Measurements Representation}
We represent images using the widely adopted \emph{bag-of-words}
model~\cite{sivic03video}. In the document modeling formulation, text documents
are represented as histograms of word counts from a given vocabulary. This model
can be easily applied to computer vision tasks, words being replaced by features
and text documents by images.

We use Scale-Invariant Feature Transform (SIFT)~\cite{lowe04distinctive}
descriptors computed at Difference of Gaussians (DoG) keypoints. SIFT
descriptors have been shown to be highly discriminative for object recognition.
Although some authors claim that they obtain better results with dense
grid~\cite{feifei05bayesian} representations, DoG interest points are more
suitable for our purpose. Indeed, we are not interested in capturing uniform
regions such as sky, but rather focused on objects.

A regular grid cuts an image into local patches on
which features are computed.  We use a patch size of $l\times l$ and
Scale-Invariant Feature Transform (SIFT) \cite{lowe04distinctive}
descriptors as they have been shown to be highly discriminative for
object recognition. $N$ images are randomly selected from the entire
dataset to build a \emph{codebook} or dictionary of features using
K-means clustering. Each patch of an image is then assigned to the
nearest \emph{codeword} of the dictionary and we can therefore build a
convenient histogram representation.

The link between bag-of-\emph{features} models in computer vision and
bag-of-words model in text modeling is intuitive. We can therefore use
the generative model of \cite{madsen05modeling} to represent an image
in a probabilistic manner as was already proposed in
\cite{ranganathan09bayesian}.  The idea of \cite{madsen05modeling} is
to represent a text document with a Dirichlet Compound Multinomial
(DCM) model, also known as Multivariate Polya distribution. It models
the fact that when a particular word occurs in a document, it is more
likely to appear again. In terms of images, this makes sense since the
same feature is likely to appear several times. A DCM is simply a
multinomial distribution with parameters $\boldsymbol{\theta}=
[\theta_1,\theta_2,\dots,\theta_K]$ which have been sampled from a
Dirichlet prior with parameters $\boldsymbol{\alpha}=
[\alpha_1,\alpha_2,\dots,\alpha_K]$. Formally, it is expressed as

\begin{equation}
\label{eqn:polya}
p(\mathbf{z}\mid \boldsymbol{\alpha}) = \int_{\boldsymbol{\theta}} p(\mathbf{z}\mid \boldsymbol{\theta})
  p(\boldsymbol{\theta}\mid \boldsymbol{\alpha})d\boldsymbol{\theta},
\end{equation}

with $p(\mathbf{z}\mid \boldsymbol{\theta})$ a multinomial distribution with
parameter $\boldsymbol{\theta}$ and $p(\boldsymbol{\theta}\mid \boldsymbol{\alpha})$
a Dirichlet distribution with parameter $\boldsymbol{\alpha}$. We notice that we
can sample from (\ref{eqn:polya}) only knowing $\boldsymbol{\alpha}$.

Performing the integration in (\ref{eqn:polya}) yields the final closed-form
solution

\begin{equation}
\label{eqn:polya_integrated}
p(\mathbf{z}\mid \boldsymbol{\alpha}) =
  \frac{\Gamma(\sum_k^K\alpha_k)}{\Gamma(\sum_k^K n_k + \alpha_k)}
  \prod_k^K\frac{\Gamma(n_k+\alpha_k)}{\Gamma(\alpha_k)},
\end{equation}

where $\Gamma(.)$ is the Gamma function and $n_k=\sum_j\;\delta(z_j-k)$.

As there exists no analytical solution to the maximum likelihood parameters
estimation of a DCM, an iterative gradient descent optimization is used and
leads to the update rule

\begin{equation}
\label{eqn:alpha_update}
\alpha_k^{new} = \alpha_k\frac{\sum_d^D\Psi(n_{dk}+\alpha_k)-\Psi(\alpha_k)}
  {\sum_d^D\Psi(n_d+\sum_{k'}^K\alpha_{k'})-\Psi(\sum_{k'}^K\alpha_{k'})},
\end{equation}

where $\Psi(.)$ is the digamma function and $D$ the number of data points
$\mathbf{z}$ used for the estimation.
