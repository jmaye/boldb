Our aim is to find a label for each segmented motion pattern. This label
represents a traffic situation, e.g., a stop or turn condition. Moreover, we are
interested in associating two different motion segments to the same label
whenever they depict the same traffic situation. In the following, we show how
we can integrate this labeling into the on-line framework of
Section~\ref{sec:motion}.

\subsection{Traffic Situation Model}
As shown above, we denote the label of a segment $r_t$ as $l_t$. This label can
take values in $\{1,2,\dots,N\}$ corresponding to $N$ parametric models $M_1,
M_2,\dots,M_N$. Each of the $M_i$ is a generative model $p(\mathbf{c}_t\mid
\boldsymbol{\eta}_i)$ for a particular traffic situation with parameter vector
$\boldsymbol{\eta}_i$. At time $t$, we estimate the distribution over the known
models conditioned on the data seen so far and the segment we are in with Bayes
law as

\begin{eqnarray}
\label{eqn:labeling}
p(l_t\mid r_t,\mathbf{c}_{1:t})&\propto&p(\mathbf{c}_t\mid l_t,r_t,
\mathbf{c}_{1:t-1})p(l_t\mid r_t,\mathbf{c}_{1:t-1})\nonumber\\
&=& p(\mathbf{c}_t\mid l_t,r_t,\mathbf{c}_{t-1})p(l_t\mid
r_t,\mathbf{c}_{1:t-1}).
\end{eqnarray}

For the prior part in \eqref{eqn:labeling}, we use the posterior of the previous
time step, that is
$p(l_t\mid r_t,\mathbf{c}_{1:t-1})=p(l_{t-1}\mid r_{t-1},\mathbf{c}_{1:t-1})$.
If we are in a new segment with $r_t=0$, we set the prior to a uniform
distribution over the known models, i.e.,
$p(l_t=1:N\mid r_t,\mathbf{c}_{1:t-1})=\frac{1}{N}$. The likelihood part in
\eqref{eqn:labeling} is computed with the model probability density function
$p(\mathbf{c}_t\mid \boldsymbol{\eta}_i)$ in the same fashion as in
\eqref{eqn:preddistr}, i.e., using a conjugate prior with hyperparameters
$\boldsymbol{\psi}_i$ as will be detailed below. Furthermore, we have only kept
the dependency on the last data point $\mathbf{c}_{t-1}$ since we update the
parameters iteratively.

As we want to be able to discover new traffic situations on-line, we have to
state if the current data $\mathbf{c}_t$ is unlikely to come from any of the
$N$ known models so far. We use Bayesian hypothesis testing and compute
the \emph{Bayes factor}~\cite{kass95bayes} for all the models $M_i$ against an
alternative model:

\begin{equation}
\label{eqn:bayesfactor}
B = \frac{p(\mathbf{c}_t\mid l_t=i,\boldsymbol{\psi}_i)}{p(\mathbf{c}_t\mid
l_t,r_t,\mathbf{c}_{t-1},\boldsymbol{\psi}^{r_{t-1}})},
\end{equation}

where $\boldsymbol{\psi}^{r_{t-1}}$ are the hyperparameters learned over the
current segment $r_{t-1}$.

The value $B$ in \eqref{eqn:bayesfactor} indicates our confidence in the model
$M_i$ and we compare it to a threshold $\xi$ for the decision. In case all
models are rejected, we create a new instance $M_{N+1}$ with hyperparameter
vector $\boldsymbol{\psi}^{r_{t-1}}$, set $p(l_t=N+1\mid
r_t,\mathbf{c}_{1:t})=p_{new}$, and $p(l_t=1:N\mid r_t,\mathbf{c}_{1:t})=
(1-p_{new})/N$. We finally update the hyperparameters $\boldsymbol{\psi}_i$ of
model $M_i$, such that $i=\argmax_{j=1:N}p(l_t=j\mid
r_t,\mathbf{c}_{1:t})$, with $\mathbf{c}_t$.

From an implementation point of view, we attach the distribution
\eqref{eqn:labeling}, the hyperparameters $\boldsymbol{\psi}^{r_{t-1}}$, and the
incremental set of known models $M_i$ to each particle. Thus, our system is able
to learn new traffic situations on-line and refine its knowledge over
previously visited scenes.

\subsection{Measurements Representation}
We represent images using the widely adopted \emph{bag-of-words}
model~\cite{sivic03video}. In the document modeling formulation, text documents
are represented as histograms of word counts from a given dictionary. This model
can be easily applied to computer vision tasks, words being replaced by features
and text documents by images.

We use Scale-Invariant Feature Transform (SIFT)~\cite{lowe04distinctive}
descriptors computed at Difference of Gaussians (DoG) keypoints. SIFT
descriptors have been shown to be highly discriminative for object recognition.
Although some authors claim that they obtain significantly better results with
dense grid representations~\cite{feifei05bayesian}, DoG interest points are more
suitable for our purpose. Indeed, we are not interested in capturing uniform
regions such as sky, but rather focused on objects. $N$ images are randomly
selected from the entire dataset to build a \emph{codebook} or dictionary of
features using K-means clustering. Each feature of an image is then assigned to
the nearest \emph{codeword} of the dictionary and we can therefore build a
convenient histogram representation.

The link between bag-of-\emph{features} models in computer vision and
bag-of-words models in text document modeling is intuitive. We can therefore use
the generative model of~\cite{madsen05modeling} to represent an image in a
probabilistic manner as was already proposed in~\cite{ranganathan09bayesian}.
Image histograms are modeled with a Dirichlet Compound Multinomial (DCM), also
known as multivariate Polya distribution. The DCM combines a multinomial model
and a Dirichlet prior, and provides an analytical solution to the
marginalization of the multinomial parameters~\cite{minka03estimating}. The
multinomial distribution $p(\mathbf{c}_t\mid\boldsymbol{\theta})$ has parameters
$\boldsymbol{\theta}=[\theta_1,\theta_2,\dots,\theta_K]$, corresponding to
$\boldsymbol{\eta}$ above. The Dirichlet prior
$p(\boldsymbol{\theta}\mid\boldsymbol{\alpha})$ has
hyperparameters $\boldsymbol{\alpha}=[\alpha_1,\alpha_2,\dots,\alpha_K]$,
corresponding to $\boldsymbol{\psi}$ above. The likelihood part of
\eqref{eqn:labeling} can now be formulated as

\begin{eqnarray}
\label{eqn:polya}
p(\mathbf{c}_t\mid l_t,r_t,\mathbf{c}_{t-1},\boldsymbol{\alpha}^{r_{t-1}})
&=&\nonumber\\
\int_{\boldsymbol{\theta}^{r_{t-1}}}
p(\mathbf{c}_{t}\mid\boldsymbol{\theta}^{r_{t-1}})
p(\boldsymbol{\theta}^{r_{t-1}}\mid
l_t,r_t,\mathbf{c}_{t-1},\boldsymbol{\alpha}^{r_{t-1}})
d\boldsymbol{\theta}^{r_{t-1}}
&=&\nonumber\\\frac{n!}{\prod_{k=1}^K n_k!}\frac{\Gamma(\alpha^{r_{t-1}})}
{\Gamma(n+\alpha^{r_{t-1}})}\prod_{k=1}^K\frac{\Gamma(n_k+\alpha^{r_{t-1}}_k)}
{\Gamma(\alpha^{r_{t-1}}_k)},
\end{eqnarray}

where $\Gamma(.)$ is the Gamma function, $n_k=\mathbf{c}_t\,(k)$,
$n=\sum_{k=1}^K n_k$,
$\alpha^{r_{t-1}}=\sum_{k=1}^K\alpha^{r_{t-1}}_k$, and we have added the
hyperparameters $\boldsymbol{\alpha}^{r_{t-1}}$ in the conditional.

For the iterative update of the hyperparameters $\boldsymbol{\alpha}^{r_t}$, we
can use the simple rule

\begin{equation}
\label{eqn:alpha_update}
\boldsymbol{\alpha}^{r_t} = \boldsymbol{\alpha}^{r_{t-1}} + \mathbf{c}_{t}.
\end{equation}

In case we start a new segment and $r_t=0$, the hyperparameters are fixed to
some prior values $\boldsymbol{\psi}_0=\{\boldsymbol{\alpha}_0\}$.
