Our aim is to find a label for each segmented motion pattern. This label
represents a traffic situation, e.g., a stop or turn condition. Moreover, we are
interested in associating two different motion segments to the same label
whenever they depict the same traffic situation. In the following, we show how
we can integrate this labeling into the on-line framework of
Section~\ref{sec:motion}.

\subsection{Traffic Situation Model}
As shown above, we denote the label of a segment $r_t$ as $l_t$. This label can
take values in $[1,2,\dots,N]$ corresponding to $N$ parametric models $M_1,M_2,
\dots,M_N$. Each of the $M_i$ is a generative model $p(\mathbf{c}_t\mid
\boldsymbol{\eta}_i)$ for a particular traffic situation with parameter vector
$\boldsymbol{\eta}_i$. At time $t$, we estimate the distribution over the known
models conditioned on the data seen so far and the segment we are in with Bayes
law as

\begin{eqnarray}
\label{eqn:labeling}
p(l_t\mid r_t,\mathbf{c}_{1:t})&\propto&p(\mathbf{c}_{1:t}\mid l_t,r_t)
p(l_t\mid r_t)\nonumber\\
&=& p(\mathbf{c}^{r_t}\mid l_t,r_t)p(l_t\mid r_t),
\end{eqnarray}

where $\mathbf{c}^{r_t}$ represents data on the current segment $r_t$, i.e.,
$\mathbf{c}^{r_{t}}=[\mathbf{c}_{t}]$ in case $r_{t}=0$ and
$\mathbf{c}^{r_{t}}=[\mathbf{c}_{t-r_{t}},\dots,\mathbf{c}_{t}]$ otherwise.

For the prior part in \eqref{eqn:labeling}, we use the posterior of the previous
time step, that is
$p(l_t\mid r_t)=p(l_{t-1}\mid r_{t-1},\mathbf{c}^{r_{t-1}})$. If we are in a new
segment with $r_t=0$, we set the prior to a uniform distribution over the known
models, i.e., $p(l_t=1:N\mid r_t)=\frac{1}{N}$. The likelihood part in
\eqref{eqn:labeling} is computed with the model probability density function
$p(\mathbf{c}_t\mid \boldsymbol{\eta}_i)$ in the same fashion as in
\eqref{eqn:preddistr}, i.e., using a conjugate prior with hyperparameters
$\boldsymbol{\psi}_i$ as will be detailed below.

As we want to be able to discover new traffic situations on-line, we have to
state if the current data $\mathbf{c}_t$ is unlikely to come from any of the
$N$ known models so far. Following the strategy of~\cite{ranganathan10pliss},
we perform $N$ likelihood ratio test at each time step. We compare model $M_i$
with a model learned over segment $r_t$. The test statistic is

\begin{equation}
\label{eqn:statistic}
D = -2\ln\frac{p(\mathbf{c}_t\mid \boldsymbol{\psi}_i)}{p(\mathbf{c}_t\mid
\boldsymbol{\psi}^{r_t})},
\end{equation}

where $\boldsymbol{\psi}^{r_t}$ is the maximum likelihood solution for
$\boldsymbol{\psi}$ over the current segment $r_t$.

Based on the value of $D$, we have to decide whether to accept or reject the
null hypothesis, i.e., data in segment $r_t$ arises from model $M_i$. It is
shown that when the sample size grows towards infinity, $D$ converges towards a
Chi-square distribution with $K-1$ degrees of freedom, where $K$ is the
dimension of $\boldsymbol{\psi}$. In our case, the sample size represents the
number of data used to compute the maximum likelihood solution. As will be
shown below, it will be sufficient to approximate $D$ with a Chi-square
distribution. We will thus compare $D$ to the Chi-square value corresponding to
the desired statistical significance $\xi$ of the null hypothesis. In practice,
this is done by computing the value of the cumulative distribution
function of the Chi-square distribution for $D$ and reject the model
$M_i$ if it is below $\xi$.

In case all models are rejected, we create a new instance $M_{N+1}$
with hyperparameter vector $\boldsymbol{\psi}^{r_t}$, set $p(l_t=N+1\mid r_t,
\mathbf{c}^{r_t})=p_{new}$, and $p(l_t=1:N\mid r_t,\mathbf{c}^{r_t})=
(1-p_{new})/N$. We update the hyperparameters $\boldsymbol{\psi}_i$ of model
$M_i$, such that $i=\argmax_{j=1:N}p(l_t=j\mid
r_t,\mathbf{c}^{r_t})$, with $\mathbf{c}_t$.

From an implementation point of view, we attach the distribution
\eqref{eqn:labeling}, the hyperparameters $\boldsymbol{\psi}^{r_t}$, and the
incremental set of known models $M_i$ to each particle. Thus, our system is able
to learn new traffic situations on-line and refine its knowledge over
previously visited scenes.

\subsection{Measurements Representation}
We represent images using the widely adopted \emph{bag-of-words}
model~\cite{sivic03video}. In the document modeling formulation, text documents
are represented as histograms of word counts from a given dictionary. This model
can be easily applied to computer vision tasks, words being replaced by features
and text documents by images.

We use Scale-Invariant Feature Transform (SIFT)~\cite{lowe04distinctive}
descriptors computed at Difference of Gaussians (DoG) keypoints. SIFT
descriptors have been shown to be highly discriminative for object recognition.
Although some authors claim that they obtain better results with dense grid
representations~\cite{feifei05bayesian}, DoG interest points are more suitable
for our purpose. Indeed, we are not interested in capturing uniform regions such
as sky, but rather focused on objects. $N$ images are randomly selected from the
entire dataset to build a \emph{codebook} or dictionary of features using
K-means clustering. Each feature of an image is then assigned to the nearest
\emph{codeword} of the dictionary and we can therefore build a convenient
histogram representation.

The link between bag-of-\emph{features} models in computer vision and
bag-of-words models in text document modeling is intuitive. We can therefore use
the generative model of~\cite{madsen05modeling} to represent an image in a
probabilistic manner as was already proposed in~\cite{ranganathan09bayesian}.
The idea of~\cite{madsen05modeling} is to represent a text document with a
Dirichlet Compound Multinomial (DCM) model, also known as multivariate Polya
distribution. It models the fact that when a particular word occurs in a
document, it is more likely to appear again. In terms of images, this makes
sense since the same feature is likely to appear several times in the same
image. Furthermore, the DCM combines a multinomial model and a Dirichlet prior,
and provides an analytical solution to the marginalization of the multinomial
parameters. The multinomial distribution $p(\mathbf{c}_t\mid
\boldsymbol{\theta})$ has parameters $\boldsymbol{\theta}=[\theta_1,\theta_2,
\dots,\theta_K]$, corresponding to $\boldsymbol{\eta}$ above. The Dirichlet
prior $p(\boldsymbol{\theta}\mid\boldsymbol{\alpha})$ has hyperparameters
$\boldsymbol{\alpha}=[\alpha_1,\alpha_2,\dots,\alpha_K]$, corresponding to
$\boldsymbol{\psi}$ above. The likelihood part of \eqref{eqn:labeling} can now
be formulated as

\begin{eqnarray}
\label{eqn:polya}
p(\mathbf{c}^{r_t}\mid l_t,r_t,\boldsymbol{\alpha}^{r_t})&=&
\int_{\boldsymbol{\theta}}
p(\mathbf{c}^{r_t}\mid\boldsymbol{\theta})p(\boldsymbol{\theta}\mid
l_t,r_t,\boldsymbol{\alpha}^{r_t})d\boldsymbol{\theta}\\\nonumber
&=&\frac{n!}{\prod_{k=1}^K n_k!}\frac{\Gamma(\alpha^{r_t})}
{\Gamma(n+\alpha^{r_t})}\prod_{k=1}^K\frac{\Gamma(n_k+\alpha^{r_t}_k)}
{\Gamma(\alpha^{r_t}_k)},
\end{eqnarray}

where $\Gamma(.)$ is the Gamma function, $n_k=\sum_{j=t-r_t}^t
\mathbf{c}_j\,(k)$, $n=\sum_{k=1}^K n_k$,
$\alpha^{r_t}=\sum_{k=1}^K\alpha^{r_t}_k$, and we added the hyperparameter
$\boldsymbol{\alpha}^{r_t}$ in the conditional.

As there exists no analytical solution to the maximum likelihood parameters
estimation of a DCM, an iterative gradient descent
optimization~\cite{minka03estimating} is used and leads to the update rule

\begin{equation}
\label{eqn:alpha_update}
\alpha_k^{r_t} = \alpha^{r_{t-1}}_k\frac{\sum_{j=t-r_t}^t\Psi(n_{jk}+
\alpha^{r_{t-1}}_k)-\Psi(\alpha^{r_{t-1}}_k)}{\sum_{j=t-r_t}^t\Psi(n_j+
\alpha^{r_{t-1}})-\Psi(\alpha^{r_{t-1}})},
\end{equation}

where $\Psi(.)$ is the digamma function, $n_{jk}=\mathbf{c}_j\,(k)$, and
$n_j=\sum_{k=1}^K n_{jk}$.
