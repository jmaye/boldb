One research area that has turned more and more into the focus of interest
during the last years is the development of driver intelligent assistant
systems. In particular, a very active topic is the design of human-friendly
vehicle control systems, able to meet the driver specific behaviors.
In this paper we address the problem of learning driver behaviors by using multiple sensing modalities, namely 
camera images and inertial information (IMU). 
Most existing techniques build a behavior model from heuristic
rules or from supervised training data. 
Our approach proposes a novel bayesian on-line and unsupervised way of learning driver's behavior in different traffic
conditions. Specifically we learn the relationship between vehicle motion and
image streams. 
% In this paper we introduce a novel
% bayesian online technique that segments IMU motion data and image streams into traffic situations and
% gives predictions for driving behaviours.
% IMU motion data via a Bayesian online learning method and labels them as different traffic situations via 
% a probablistic Dirichlet Compound Multinomial (DCM) modeling of features in image streams.
% The online learned model is used for predicting driving actions given IMU and camera data.
% 
% 
% makes use of sound statistical methods to segment 
% IMU data and image streams into driving behaviours. 
% Furthermore, we are able to use such model 
% for predicting driving actions.
% 
% 
% a \textit{change-point} detection method to segment IMU 
% data into portions 
% 
% and
% image streams into portions corresponding to behaviors.
Motion data is segmented by using a \textit{change-point} detection method, a technique that solves
 the problem of detecting abrupt changes to the parameters of a statistical
model.
%  This is computed on-line and in an unsupervised manner. 
% The motion data
% segmented by these changes define a driver behavior.
We use a Bayesian approach to compute the probability of a
change-point occurring at each time step. 
% A change point is obtained by
% combining a prior on the occurrence of change-points with the likelihood of the
% current measurement given all the possible scenarios in which change-points
% could have occurred in the past. 
% 
We have formulated a
fast approximation of this method by using a Rao-Blackwellized particle
filter.  The motion segments are grouped togheter to represent traffic situations by 
exploiting similarity in the associated image streams.  This is computed on-line and in an unsupervised manner.
The streams are represented as
collection of bag-of-words modeled after a Dirichlet Compound Multinomial model. 
This process provides tools to predict the driving 
actions conditioned on the current traffic situation. 
% 
% Furthermore, each particle learns a distribution over the known traffic
% situations and possible actions. Traffic situations and action models are
% updated on-line. For each time step, we are thus able to infer the most probable
% action to apply given a traffic situation.

The the paper is structured as follows. Section~\ref{sec:related}
summarizes the previous work related to ours. Section~\ref{sec:motion}
describes our motion segmentation method. Section~\ref{sec:labeling}
shows how we model a traffic situation. Section~\ref{sec:action} demonstrates
our action model. Section~\ref{sec:exp} presents experimental results.
Section~\ref{sec:conc} outlines our conclusions and provides some insights for
future work.
