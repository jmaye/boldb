One research area that has turned more and more into the focus of interest
during the last years is the development of driver intelligent assistant
systems. In particular, a very active topic is the design of human-friendly
vehicle control systems, able to meet the driver specific behaviors.
In this paper we address the problem of learning driver behaviors based on
camera images and inertial information (IMU). Our approach proposes a novel
on-line and unsupervised way of learning driver's behavior in different traffic
conditions. Specifically we learn the relationship between vehicle motion and
image streams. Most existing techniques build a behavior model from heuristic
rules or from supervised training data. In this paper we introduce a novel
technique that uses a \textit{change-point} detection model to segment IMU and
image streams into portions corresponding to behaviors. Change-point detection
is the problem of detecting abrupt changes to the parameters of a statistical
model. This is computed on-line and in an unsupervised manner. The motion data
segmented by these changes define a driver behavior.

The idea is to use a Bayesian approach to compute the probability of a
change-point occurring at each time step. A change point is obtained by
combining a prior on the occurrence of change-points with the likelihood of the
current measurement given all the possible scenarios in which change-points
could have occurred in the past. We have formulated a
practical approximation of this method by using a Rao-Blackwellized particle
filter. Furthermore, each particle learns a distribution over the known traffic
situations and possible actions. Traffic situations and action models are
updated on-line. For each time step, we are thus able to infer the most probable
action to apply given a traffic situation.

The rest of the paper is structured as follows. Section~\ref{sec:related}
summarizes the previous work related to ours. Section~\ref{sec:motion}
describes our motion segmentation method. Section~\ref{sec:labeling}
shows how we model a traffic situation. Section~\ref{sec:action} demonstrates
our action model. Section~\ref{sec:exp} presents experimental results.
Section~\ref{sec:conc} outlines our conclusions and provides some insights for
future work.
